\documentclass[12pt]{article}
\usepackage{graphicx}
\usepackage{float}
\usepackage{subcaption}
\usepackage{hyperref}
\usepackage{mathtools}
\usepackage[usenames,dvipsnames]{xcolor}
\usepackage[authoryear]{natbib}
\usepackage{amsmath}
\usepackage{amsfonts}
\usepackage{bigints}
\usepackage{array}
\usepackage{geometry}

\hypersetup{
    colorlinks,
    linkcolor={red!50!black},
    citecolor={blue!50!black},
    urlcolor={blue!80!black}
}

\setlength{\parskip}{\baselineskip}
\setlength{\parindent}{0pt}

\newcommand{\e}[1]{{\mathbb E}\left[ #1 \right]}
\newcommand{\var}[1]{{\mathbb V \textnormal{ar}}\left[ #1 \right]}
\newcommand{\gb}[1]{{\it\color{ForestGreen}{(#1)}}}
\newcommand{\plr}[1]{{\it\color{RedViolet}{(#1)}}}
\newcommand{\gc}[1]{{\it\color{blue}{(#1)}}}

\geometry{a4paper}

\title{Inference of Continuous and Discrete Population Structure Through Space and Time}
\date{\vspace{-5ex}}
\author{Gideon S. Bradburd$^{1,a}$, Peter L. Ralph$^{2,b}$, Graham M. Coop$^{1,c}$}

\begin{document}

\maketitle

\textsuperscript{1}Center for Population Biology, Department of Evolution and Ecology, University of California, Davis, CA 95616

\textsuperscript{2}Department of Molecular and Computational Biology, University of Southern California, Los Angeles, CA 90089

\textsuperscript{a}gbradburd@ucdavis.edu; 
\textsuperscript{b}pralph@usc.edu;
\textsuperscript{c}gmcoop@ucdavis.edu\\\\\

\newpage

\section{Model}
Ok so we have $N$ individuals sampled throughout time and space, 
and, for each individual, we have the genotype at each of $L$ loci.
We are interested in modeling both continuous patterns of population differentiation 
(isolation across space and over time), as well as discrete population structure.
In space, discrete structure might be due to a barrier to dispersal, 
or to a recent expansion event that has brought relatively differentiated groups into geographic contact.
In time, discrete structure might correspond to population replacement, 
or the sudden influx of gene flow from somewhere not nearby.

To model these dual patterns of structure (continuous in time and space, as well as discrete), 
we say that there are $K$ populations in space-time.  
Within each population (which are roughly analogous to the clusters in STRUCTURE),
covariance decays continuously with spatial and temporal distance. 
The spatiotemporal covariance between individuals $i$ and $k$ takes the following form in population $k$:
\begin{equation}
\label{eq:pop_cov}
F^{(k)}_{i,j} = \alpha^{(k)}_0 \text{exp} \left(	  \left( \frac{D_{i,j}}{\alpha^{(k)}_D} + \frac{\tau_{i,j}}{\alpha^{(k)}_{\tau}} \right) ^{\alpha^{(k)}_2} \right)
\end{equation}

\gb{Peter, this form of covariance is a placeholder.  
I've been looking into different forms from Gneiting 2002, 
but would love some feedback on considerations 
when choosing which properties of which Matern functions are desirable}.

Individuals are then admixed between those populations, 
choosing membership $w^{(k)}$ in population $k$, 
where $w^{(k)}$ is the probability that a sampled allele came from population $k$.

We model the covariance in allele frequencies, $\Omega$, across loci between individuals $i$ and $j$
as a sum of their spatiotemporal covariances within each cluster, weighted by their
membership proportions in each cluster.

\begin{equation}
\label{eq:total_cov}
\Omega_{i,j} = \sum\limits_K \left(w_i^{(k)}w_j^{(k)} \left( F^{(k)}_{i,j} + \mu^{(k)} \right)\right) + \delta
\end{equation}

Here, $\mu^{(k)}$ is the population-level effect of sharing some deviation away from an ancestral allele frequency, 
and $delta$ is the global effect of sharing an ancestral allele frequency.

\gb{Peter, our thoughts proceed as follows:\\
say $X_i$ is the allele frequency at a given locus in population $i$,\\
and $\epsilon$ is the ancestral frequency at that locus,\\
and $\Delta X_i$ is the deviation in population $i$ from $\epsilon$ at that locus:\\
\begin{align}
\text{Cov}(f_i,f_j) &= \e{X_iX_j} - \e{X_i}\e{X_j}	\notag\\
			&= \e{(\Delta X_i + \epsilon) (\Delta X_j + \epsilon)} - \e{X_i+ \epsilon}\e{X_j+ \epsilon} \notag\\
			&= \e{(\Delta X_i \Delta X_j + \Delta X_i\epsilon + \Delta X_j\epsilon + \epsilon^2)} -(\e{X_i}+ \e{\epsilon})(\e{X_j}+ \e{\epsilon}) \notag\\
			&= \e{\Delta X_i \Delta X_j}  + \e{\epsilon^2} - \e{\epsilon}^2 \notag\\
			&= \e{\Delta X_i \Delta X_j}  + \var{\epsilon}
\end{align}
\\
So in Eqn. \eqref{eq:total_cov}, the $\delta$ term is describing this $\var{\epsilon}$.
If this math works, then it also means that we don't have to mean-center the loci;
we can just model covariance in sample frequencies, 
and add a single parameter to describe the increased covariance between samples across loci 
due to the ancestral mean they share at each locus.}















\end{document}