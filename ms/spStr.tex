\documentclass[12pt]{article}
\usepackage{graphicx}
\usepackage{float}
\usepackage{subcaption}
\usepackage{hyperref}
\usepackage{mathtools}
\usepackage[usenames,dvipsnames]{xcolor}
\usepackage[authoryear]{natbib}
\usepackage{amsmath}
\usepackage{amsfonts}
\usepackage{bigints}
\usepackage{array}
\usepackage{tikz}
\usepackage{longtable}
\usepackage{geometry}


\bibliographystyle{plainnat}

\setlength{\parskip}{\baselineskip}
\setlength{\parindent}{0pt}

\newcommand{\e}[1]{{\mathbb E}\left[ #1 \right]}
\newcommand{\given}{\mid}
\newcommand{\secref}[1]{``\nameref{#1}''}


\newcommand{\gb}[1]{{\it\color{magenta}{(#1)}}}
\newcommand{\plr}[1]{{\it\color{purple}{(#1)}}}
\newcommand{\gc}[1]{{\it\color{blue}{(#1)}}}

\geometry{a4paper}

\title{Inferring Continuous and Discrete Population Genetic Structure Across Space and Time}
\date{\vspace{-5ex}}
\author{
Gideon S. Bradburd$^{1,a}$, 
Peter L. Ralph$^{2,b}$, 
Graham M. Coop$^{3,c}$}

\begin{document}

\maketitle

\textsuperscript{1}
Museum of Vertebrate Zoology, 
Department of Environmental Science, Policy, and Management, 
University of California, Berkeley, CA 94720

\textsuperscript{2}
Department of Molecular and Computational Biology, 
University of Southern California, Los Angeles, CA 90089

\textsuperscript{3}
Center for Population Biology,
 Department of Evolution and Ecology, 
 University of California, Davis, CA 95616

\textsuperscript{a}bradburd@berkeley.edu; 
\textsuperscript{b}pralph@usc.edu;
\textsuperscript{c}gmcoop@ucdavis.edu\\\\\

\newpage
 

\begin{abstract}
One of the classic problems in population genetics is the characterization 
of discrete population structure when the genotyped samples also show 
continuous patterns of genetic differentiation.
Especially when sampling is discontinuous, 
clustering or assignment methods may incorrectly ascribe differentiation 
due to continuous processes (e.g., drift across space or time) 
to discrete processes, such as geographic, reproductive, or behavioral barriers 
between populations.
This is partly a result of the difficulty of sampling uniformly and continuously 
across the spatiotemporal range of a population or species, 
but more, it reflects a shortcoming of current methods for inferring and 
visualizing population structure from genetic data in the face of data 
that are characterized by both continuous and discrete population structure.
Here, we present a novel statistical framework for the simultaneous inference 
of continuous and discrete patterns of population structure.
The method estimates ancestry proportions for each 
sample from a set of discrete population clusters, 
and, within each cluster, estimates a rate at which relatedness decays with 
spatial and/or temporal distance.
This model explicitly addresses the ``clines vs. clusters" problem in 
quantifying population structure by jointly accommodating both 
continuous and discrete patterns of differentiation. 
The model also naturally captures population replacement, 
a phenomenon for which there is substantial evidence in humans 
from archaeological evidence and ancient DNA. 
We demonstrate the utility of this approach using a combination of 
ancient and modern human individuals sampled throughout Europe, 
and find evidence for aliens.
\end{abstract}

%evolutionary history of the genotyped samples has also been shaped by geographically continuous processes, 
%such as migration.


\newpage

\section*{Introduction}

\section*{Methods}

\section*{Results}

\subsection*{Simulations}

\subsection*{Empirical Applications}

\section*{Discussion}

\section*{Acknowledgements}

This work was supported in part by 
the National Science Foundation under award number NSF \#1262645 (DBI) to PR and GC, 
the National Institute of General Medical Sciences of the National Institutes of Health under award numbers NIH RO1GM83098 and RO1GM107374 to GC,
and the National Science Foundation under award numbers NSF \# 1148897 and \# 1402725 to GB.


\newpage

%\bibliography{../spacemix.refs}

\newpage

\pagenumbering{gobble}
\section*{Supplementary Materials}
\renewcommand{\thefigure}{S\arabic{figure}}
\setcounter{figure}{0}
\renewcommand{\thetable}{S\arabic{table}}
\setcounter{table}{0}
\renewcommand{\theequation}{S\arabic{table}}
\setcounter{equation}{0}

\end{document}
