\documentclass[12pt]{article}
\usepackage{graphicx}
\usepackage{float}
\usepackage{subcaption}
\usepackage{hyperref}
\usepackage{mathtools}
\usepackage[usenames,dvipsnames]{xcolor}
\usepackage[authoryear]{natbib}
\usepackage{amsmath}
\usepackage{amsfonts}
\usepackage{bigints}
\usepackage{array}
\usepackage{tikz}
\usepackage{longtable}
\usepackage{geometry}

\bibliographystyle{plainnat}

\setlength{\parskip}{\baselineskip}
\setlength{\parindent}{0pt}

\newcommand{\e}[1]{{\mathbb E}\left[ #1 \right]}
\newcommand{\given}{\mid}
\newcommand{\secref}[1]{``\nameref{#1}''}


\newcommand{\gb}[1]{{\it\color{magenta}{(#1)}}}
\newcommand{\plr}[1]{{\it\color{purple}{(#1)}}}
\newcommand{\gc}[1]{{\it\color{blue}{(#1)}}}

\geometry{a4paper}

\title{Inferring Continuous and Discrete Population Genetic Structure Across Space and Time}
\date{\vspace{-5ex}}
\author{
Gideon S. Bradburd$^{1,a}$, 
Peter L. Ralph$^{2,b}$, 
Graham M. Coop$^{3,c}$}

\begin{document}

\maketitle

\textsuperscript{1}
Museum of Vertebrate Zoology, 
Department of Environmental Science, Policy, and Management, 
University of California, Berkeley, CA 94720

\textsuperscript{2}
Department of Molecular and Computational Biology, 
University of Southern California, Los Angeles, CA 90089

\textsuperscript{3}
Center for Population Biology,
 Department of Evolution and Ecology, 
 University of California, Davis, CA 95616

\textsuperscript{a}bradburd@berkeley.edu; 
\textsuperscript{b}pralph@usc.edu;
\textsuperscript{c}gmcoop@ucdavis.edu\\\\\

\newpage
 

\begin{abstract}
One of the classic problems in population genetics is the characterization 
of discrete population structure when the genotyped samples also show 
continuous patterns of genetic differentiation.
Especially when sampling is discontinuous, 
clustering or assignment methods may incorrectly ascribe differentiation 
due to continuous processes (e.g., drift across space or time) 
to discrete processes, such as geographic, reproductive, or behavioral barriers 
between populations.
This is partly a result of the difficulty of sampling uniformly and continuously 
across the spatiotemporal range of a population or species, 
but more, it reflects a shortcoming of current methods for inferring and 
visualizing population structure from genetic data in the face of data 
that are characterized by both continuous and discrete population structure.
Here, we present a novel statistical framework for the simultaneous inference 
of continuous and discrete patterns of population structure.
The method estimates ancestry proportions for each 
sample from a set of discrete population clusters, 
and, within each cluster, estimates a rate at which relatedness decays with 
spatial and/or temporal distance.
This model explicitly addresses the ``clines vs. clusters" problem in 
quantifying population structure by jointly accommodating both 
continuous and discrete patterns of differentiation. 
The model also naturally captures population replacement, 
a phenomenon for which there is substantial evidence in humans 
from archaeological evidence and ancient DNA. 
We demonstrate the utility of this approach using a combination of 
ancient and modern human individuals sampled throughout Europe, 
and find compelling evidence for space aliens.
\end{abstract}

%evolutionary history of the genotyped samples has also been shaped by geographically continuous processes, 
%such as migration.


\newpage
\section*{Argument}
Patterns of population genetic differentiation are characterized by both discrete and 
continuous structure.  Identification of genetically discrete populations is important 
not just for understanding the distribution of genetic variation on a landscape, but also 
for understanding the demographic history of a species, or, from a conservation perspective, 
for characterizing discrete management units.  However, it is frequently difficult to disentangle 
continuous and discrete patterns of genetic differentiation.  

This difficulty arises for two reasons. First, both patterns are frequently present in empirical 
systems, and, with incomplete spatial or temporal sampling, it can be difficult to differentiate 
continuous and discrete patterns.  For example, if a species is dispersal limited, it may exhibit 
a pattern of isolation by distance (IBD), in which samples taken from populations separated 
by greater geographic distance are more genetically differentiated than samples taken from 
neighboring populations.  If only two samples are taken from that species, from a pair of 
populations separated by a large geographic distance, it would be impossible to determine 
whether the differentiation between them conformed to a continuous pattern, or some discrete 
pattern (perhaps due to a physical barrier to dispersal between them, or a behavioral or 
phenological or  barrier to reproduction between them).

The second reason it may be difficult to disentangle continuous and discrete patterns of 
population genetic differentiation is that such a classification (``continuous" vs. ``discrete") 
is artificial, and can rarely be applied cleanly in empirical systems.  Even a group of isolated  
populations cannot be considered truly discrete units, as they will have diverged from 
shared ancestral populations, and stopped exchanging migrants, at different points in 
the past, and should therefore be characterized along a continuum of pairwise genetic relatedness,
even though they represent `discrete' evolutionary or management units.

However, there are many empirical systems in which the different, overlaid patterns of genetic 
differentiation can be well characterized as continuous or discrete, and, in such systems, it is 
vital to model both patterns simultaneously, to avoid confusing one for the other.  Currently, 
there are no methods that can probabilistically place individuals into discrete population clusters 
while accounting for a background rate at which genetic differentiation accrues with distance 
between samples in space or time.  This gap in methods for the study of population genetic 
structure frequently results in the mischaracterization of continuous differentiation as discrete 
(and vice versa), potentially inflating the number of inferred discrete population clusters 
above what is biologically accurate.  By introducing a statistical framework that simultaneously 
models discrete and continuous patterns of population structure, we can do a more accurate 
job describing and quantifying both.


%However, many empirical systems fall into a `sweet spot,' in which the magnitude of continuous 
%genetic differentiation (e.g., IBD or drift within a population through time) is small relative to the 
%scale of discrete differentiation, and the latter is of biological or conservation interest.  These might 
%include, for example, a standing hybrid zone, or the recent removal of a barrier to gene flow between 
%populations, potentially do to an introduction or colonization event.
%
%In such `sweet spot' empirical systems, 


\newpage
\section*{Introduction}

\section*{Methods}
\paragraph{Data}
The statistical framework of our approach is conceptually similar to \cite{Wasser2004} and \cite{spacemix}.
The genetic data modeled consist of allele frequencies at $L$ unlinked, bi-allelic single nucleotide polymorphisms (SNPs) genotyped across $K$ samples.
The sample frequency at locus $\ell$ in sample $k$, $f_{\ell,k}$, is calculated by first arbitrarily choosing an allele segregating 
at locus $\ell$ to count, 
then dividing the total number of observations of that counted allele by the total number of chromosomes genotyped at that locus
in sample $k$.
Following \cite{spacemix}, 
we standardize allele frequencies by subtracting off the mean and normalizing the variance. 
Specifically, we compute the standardized allele frequency at locus $\ell$ in sample $k$, $X_{\ell,k}$, as follows:
\begin{equation}
\hat{X}_{\ell,k} = (\hat{f}_{k,\ell}  - \bar{f}_{\ell})/\sqrt{\bar{f}'_{\ell}(1-\bar{f}'_{\ell})}\text{,}
\end{equation}

where $\hat{f}_{\ell,k}$ is the sample allele frequency at locus $\ell$ in sample $k$,
$\bar{f}_{\ell}$ is the mean allele frequency across all samples at locus $\ell$.
With $\bar{f}'_{\ell}$ we make a slight adjustment to the mean to avoid extreme values:

\begin{align}
\bar{f}_{\ell} &= \frac{1}{K}\sum\limits_{K}\hat{f}_{\ell,k}\\
\bar{f}'_{\ell} &= \frac{1}{K+1}\left(\sum\limits_{K}\hat{f}_{\ell,k}+ 0.5\right)
\end{align}

This standardization procedure gives each locus roughly unit variance, 
and makes the result invariant to which allele (major or minor) is chosen to be counted at each locus.
We then calculate the sample standardized allele frequency covariance between all samples:
\begin{equation}
\widehat{\Omega} = \frac{1}{L}  \hat{X}\hat{X}^T
\end{equation}

\paragraph{Continuous and discrete differentiation}
We describe the observed patterns of genetic variation as a combination of discrete population clusters,
within which genetic variation is continuously distributed, 
and between which sampled individuals or populations can draw admixture.
Each of these discrete clusters is modeled as a spatial process
in which migration between neighboring demes acts 
to homogenize allele frequency changes that arise locally due to drift;
this interplay between drift and migration 
results in a continuous pattern of isolation by distance within each cluster.
Multiple population cluster processes can co-occur in space, 
leading to discrete jumps in allele frequencies across many loci over small geographic distances,
and the genotyped samples can be admixed between different clusters.

The continuous decay of allele frequency covariance with distance 
within a cluster is described using a simple and flexible powered exponential function.
The covariance in cluster $k$ between samples $i$ and $j$ is given by:
\begin{equation}
F^{(k)}_{i,j} = \alpha_0 \;\text{exp} \left( -(\alpha_D D_{i,j}) ^ \alpha_2	\right)
\end{equation}


\section*{Results}

\subsection*{Simulations}

\subsection*{Empirical Applications}

\section*{Discussion}

\section*{Acknowledgements}

This work was supported in part by 
the National Science Foundation under award number NSF \#1262645 (DBI) to PR and GC, 
the National Institute of General Medical Sciences of the National Institutes of Health under award numbers NIH RO1GM83098 and RO1GM107374 to GC,
and the National Science Foundation under award numbers NSF \# 1148897 and \# 1402725 to GB.


\newpage

\bibliography{reference.library.tex}

\newpage

\pagenumbering{gobble}
\section*{Supplementary Materials}
\renewcommand{\thefigure}{S\arabic{figure}}
\setcounter{figure}{0}
\renewcommand{\thetable}{S\arabic{table}}
\setcounter{table}{0}
\renewcommand{\theequation}{S\arabic{table}}
\setcounter{equation}{0}

\end{document}
